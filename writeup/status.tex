%% LyX 2.3.3 created this file.  For more info, see http://www.lyx.org/.
%% Do not edit unless you really know what you are doing.
\documentclass[english]{article}
\usepackage[T1]{fontenc}
\usepackage[utf8]{inputenc}
\usepackage{float}
\usepackage{microtype}

\makeatletter
%%%%%%%%%%%%%%%%%%%%%%%%%%%%%% User specified LaTeX commands.
\usepackage[final]{nips_2016}

% to compile a camera-ready version, add the [final] option, e.g.:
% \usepackage[final]{nips_2016}

%\usepackage[utf8]{inputenc} % allow utf-8 input
%\usepackage[T1]{fontenc}    % use 8-bit T1 fonts
\usepackage{hyperref}       % hyperlinks
\usepackage{url}            % simple URL typesetting
\usepackage{booktabs}       % professional-quality tables
\usepackage{amsfonts}       % blackboard math symbols
\usepackage{nicefrac}       % compact symbols for 1/2, etc.
\usepackage{smartdiagram}

\usepackage{mathbbol}
 
\usepackage{listings}
\usepackage{xcolor}
 
\definecolor{codegreen}{rgb}{0,0.6,0}
\definecolor{codegray}{rgb}{0.5,0.5,0.5}
\definecolor{codepurple}{rgb}{0.58,0,0.82}
\definecolor{backcolour}{rgb}{0.95,0.95,0.92}
 
\lstdefinestyle{mystyle}{
    backgroundcolor=\color{backcolour},   
    commentstyle=\color{codegreen},
    keywordstyle=\color{magenta},
    numberstyle=\tiny\color{codegray},
    stringstyle=\color{codepurple},
    basicstyle=\ttfamily\footnotesize,
    breakatwhitespace=false,         
    breaklines=true,                 
    captionpos=b,                    
    keepspaces=true,                 
    numbers=left,                    
    numbersep=5pt,                  
    showspaces=false,                
    showstringspaces=false,
    showtabs=false,                  
    tabsize=2
}
 
\lstset{style=mystyle}

\makeatother

\usepackage{babel}
\usepackage[style=numeric]{biblatex}
\addbibresource{project1.bib}
\begin{document}
\title{AA222 Final Project status: Trajectory Optimization with dynamic obstacles
avoidance}

\maketitle
The project deals with trajectory optimization and control command
of an object along a path where we have to avoid obstacles crossing
our path. It is a multi-objective optimization problem: we have to
optimize efficiency (time to goal), comfort (minimizing acceleration
variations) and safety (avoiding collision). 

So we far we did:
\begin{enumerate}
\item Literature review and project objectives definition (cf below)
\item github repository setup 
\item In the continuity of a previous project , we have a first baseline
implementation. A MPC solution based on Julia JuMP.jl and ipopt.jl
which is fully functional. 
\end{enumerate}
We rely on a direct collocation method where MPC is used: we plan
over 20 time steps, for 20 commands, but apply only the first command
before taking into account new observations for re-planning. What
we plan to do next is:
\begin{enumerate}
\item Setup of our own optimizer, without relying on external packages like
JuMP and ipopt. We will leverage on project2 work where an Interior
Point Method based on log barrier was implemented \cite{10.5555/3351864}.
We implemented a first order (Conjugate Gradient) and second order
(quasi-newton BFGS) optimizer with backtracking line search.
\item Handle an inconsistent constraint of the form $abs(.)$ as per \cite{10.5555/1734063}:
\begin{enumerate}
\item With slack variables
\item With a smooth version of abs()
\end{enumerate}
\item Study the influence of various initialization strategies \cite{10.5555/1734063}
:
\begin{enumerate}
\item Initialization with a trajectory that is dynamically feasible but
not safe
\item Initialization with a Neural Network heuristic. Based on a previous
project we already have a Neural Network that can propose a candidate
solution. This solution is unsafe in 20\% of the cases but could be
used as a useful initializer.
\end{enumerate}
\item If possible, depending on fast we progress with the 3 previous objectives,
we will compare optimization over station \cite{7225840,7353382,7535558}
with optimization over time \cite{7995713} for 2D paths. In the first
case we fix the time steps and derive positions at every time step,
while in the second case we fix the spatial positions and derive the
time instants at which we go over these spatial positions. In the
first case, when dealing with 2D paths, an analytical representation
of the path is required while in the second case, a sequence of waypoints
is sufficient. But the constraints may be even more non linear and
non convex in the later case. 
\end{enumerate}
Point 2) issue is avoided in many publications by claiming that a
higher level planner decides whether we have to proceed or yield the
way. But while this may be a reasonable strategy when dealing with
a single object crossing our path, it does not scale to multiple crossing
objects. A more principled approach is required to scale with the
complexity of the scene. Point 3) could be a promising combination
of techniques: leveraging on offline Neural Network training to speed
up online optimization convergence. For Point 4) while some publications
focus on one or the other approach, we would like to outline more
precisely the pros and cons of these approaches and benchmark their
accuracy and convergence speed. 

\nocite{*}
\printbibliography

\end{document}
